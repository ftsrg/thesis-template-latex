%----------------------------------------------------------------------------
\chapter{\LaTeX-eszközök}\label{sect:LatexTools}
%----------------------------------------------------------------------------
\section{A szerkesztéshez használatos, Windows alapú eszközök}
%----------------------------------------------------------------------------
Ez a sablon Windows operációs rendszer alatt készült TeXnicCenter 1 Beta 7.01 szerkesztővel. A TeXnicCenter egy \LaTeX-szerkesztőprogram számtalan hasznos -- és ráadásul jól működő -- szolgáltatással (\figref{TexnicCenter} ábra). A szoftver ingyenesen letölthető a\\\url{http://www.texniccenter.org/} címről.

\begin{figure}[!ht]
\centering
\includegraphics[width=150mm, keepaspectratio]{figures/TeXnicCenter.png}
\caption{A TeXnicCenter Windows alapú \LaTeX-szerkesztő.} 
\label{fig:TexnicCenter}
\end{figure}

Egy másik használható Windows alapú szerkesztőprogram a LEd (LaTeX Editor,\\\url{http://www.latexeditor.org/}), a TeXnicCenter azonban stabilabb, gyorsabb, és jobban használható.

%----------------------------------------------------------------------------
\section{A dokumentum lefordítása Windows alatt}
%----------------------------------------------------------------------------
A TeXnicCenter és a LEd kizárólag szerkesztőprogram (bár az utóbbiban DVI-nézegető is van), így a dokumentum fordításához szükséges eszközöket nem tartalmazza. Windows alatt alapvetően két lehetőség közül érdemes választani: MiKTeX (\url{http://miktex.org/}) és TeXLive (\url{http://www.tug.org/texlive/}) programcsomag. Az utóbbi működik Mac OS X, GNU/Linux alatt és Unix-származékokon is. A MiKTeX egy alapcsomag telepítése után mindig letölti a használt funkciókhoz szükséges, de lokálisan hiányzó \TeX-csomagokat, míg a TeXLive DVD ISO verzóban férhető hozzá. Ez a dokumentum TeXLive 2008 programcsomag segítségével fordult, amelynek DVD ISO verziója a megadott oldalról letölthető. A sablon lefordításához a disztribúcióban szereplő \verb+magyar.ldf+ fájlt a \verb+http://www.math.bme.hu/latex/+ változatra kell cserélni, vagy az utóbbi változatot be kell másolni a projekt-könyvtárba (ahogy ezt meg is tettük a sablonban) különben anomáliák tapasztalhatók a dokumentumban (pl. az ábra- és táblázat-aláírások formátuma nem a beállított lesz, vagy bizonyos oldalakon megjelenik alapételmezésben egy fejléc). A TeXLive 2008-at még nem kell külön telepíteni a gépre, elegendő DVD-ről (vagy az ISO fájlból közvetlenül, pl. DaemonTools-szal) használni. 

A \TeX-eszközöket tartalmazó programcsomag binárisainak elérési útját minden esetben be kell állítani a szerkesztőprogramban, például TeXnicCenter esetén legegyszerűbben a \verb+Build / Define output profiles...+ menüponttal előhívott dialógusablakban a \verb+Wizard...+ gombra kattintva tehetjük ezt meg.

A PDF-\LaTeX~használata esetén a generált dokumentum közvetlenül PDF-formátumban áll rendelkezésre. Amennyiben a PDF-fájl egy PDF-nézőben (pl. Adobe Acrobat Reader vagy Foxit PDF Reader) meg van nyitva, akkor a fájlleírót a PDF-néző program tipikusan lefoglalja. Ilyen esetben a dokumentum újrafordítása hibaüzenettel kilép. Ha bezárjuk és újra megnyitjuk a PDF dokumentumot, akkor pedig a PDF-nézők többsége az első oldalon nyitja meg a dokumentumot, nem a legutóbb olvasott oldalon. Ezzel szemben például az egyszerű és ingyenes \textcolor{blue}{Sumatra PDF} nevű program képes arra, hogy a megnyitott dokumentum megváltozását detektálja, és frissítse a nézetet az aktuális oldal megtartásával.

%----------------------------------------------------------------------------
\section{Eszközök Linuxhoz}
%----------------------------------------------------------------------------
Linux operációs rendszer alatt is rengeteg szerkesztőprogram van, pl. a KDE alapú Kile jól használható. Ez ingyenesen letölthető, vagy éppenséggel az adott Linux-disztribúció eleve tartalmazza, ahogyan a dokumentum fordításához szükséges csomagokat is. Az Ubuntu Linux disztribúciók alatt például legtöbbször a \verb+texlive-base+ csomag telepítésével használhatók a \LaTeX-eszközök.
